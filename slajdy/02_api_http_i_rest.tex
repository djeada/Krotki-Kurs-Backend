\documentclass[10pt,compress,usenames,dvipsnames,aspectratio=169]{beamer}

% Ulepszony, estetyczny ciemny motyw
\usetheme{CambridgeUS}
\useoutertheme{shadow}
\usecolortheme{owl}

% Ustawienia dla polskich znaków
\usepackage[T1]{fontenc}
\usepackage[utf8]{inputenc}
\usepackage[polish]{babel}

% Usunięcie jasnego nagłówka
\setbeamertemplate{headline}{}
\setbeamertemplate{footline}{}

% Tło oraz kolory tekstu
\setbeamercolor{background canvas}{bg=black!95}
\setbeamercolor{normal text}{fg=white}
\setbeamercolor{frametitle}{bg=black!85, fg=white}
\setbeamercolor{framesubtitle}{fg=gray!60}
\setbeamercolor{title}{fg=white}
\setbeamercolor{subtitle}{fg=gray!60}
\setbeamercolor{author}{fg=white}
\setbeamercolor{institute}{fg=white}
\setbeamercolor{date}{fg=white}
\setbeamercolor{itemize item}{fg=cyan}
\setbeamercolor{enumerate item}{fg=cyan}

% Dodatkowe kolory dla czytelności
\definecolor{bgcode}{RGB}{30,30,30}
\definecolor{fgcode}{RGB}{220,220,220}
\definecolor{keyword}{RGB}{86,156,214}
\definecolor{comment}{RGB}{87,166,74}
\definecolor{string}{RGB}{214,157,133}

% Konfiguracja pakietu listings do podświetlania składni
\usepackage{listings}
\lstdefinelanguage{JavaScript}{
    morekeywords={break,case,catch,continue,
        debugger,default,delete,do,else,false,
        finally,for,function,if,in,instanceof,new,
        null,return,switch,this,throw,true,try,
        typeof,var,void,while,with},
    sensitive=true,
    morecomment=[l]{//},
    morecomment=[s]{/*}{*/},
    morestring=[b]",
    morestring=[b]',
    moredelim=[l][\color{keyword}]{.then},
    moredelim=[l][\color{keyword}]{=>}
}

% Teraz można wczytać: 
\lstloadlanguages{bash, Python, JavaScript}

\lstdefinestyle{dark}{
  backgroundcolor=\color{bgcode},
  basicstyle=\ttfamily\small\color{fgcode},
  keywordstyle=\color{keyword}\bfseries,
  commentstyle=\color{comment}\itshape,
  stringstyle=\color{string},
  showstringspaces=false,
  frame=single,
  rulecolor=\color{gray!70},
  numbers=left,
  numberstyle=\tiny\color{gray!70},
  breaklines=true,
  breakatwhitespace=true,
  tabsize=2,
  literate={ą}{{\k{a}}}1 {Ą}{{\k{A}}}1
           {ć}{{\'{c}}}1 {Ć}{{\'{C}}}1
           {ę}{{\k{e}}}1 {Ę}{{\k{E}}}1
           {ł}{{\l}}1 {Ł}{{\L}}1
           {ń}{{\'{n}}}1 {Ń}{{\'{N}}}1
           {ó}{{\'{o}}}1 {Ó}{{\'{O}}}1
           {ś}{{\'{s}}}1 {Ś}{{\'{S}}}1
           {ż}{{\.{z}}}1 {Ż}{{\.{Z}}}1
           {ź}{{\'{z}}}1 {Ź}{{\'{Z}}}1
}

\title{API, HTTP i REST}
\subtitle{Wprowadzenie do komunikacji sieciowej}
\author{Adam Djellouli}
\date{\today}

\begin{document}

% --- Slajd tytułowy (bez nagłówka/stopki) ---
\begin{frame}[plain]
  \titlepage
\end{frame}

% --- Pierwszy slajd z agendą / co będzie wiedzieć ---
\begin{frame}[fragile]{Czego się dowiesz?}
  \begin{itemize}
    \item Czym jest protokół HTTP, jak wygląda wymiana żądania i odpowiedzi oraz metody \texttt{GET}, \texttt{POST}, \texttt{DELETE} i \texttt{PUT} wraz z kodami statusu.
    \item Czym jest API, wprowadzenie do architektury REST oraz inne modele komunikacji.
    \item Jak korzystać z API przy użyciu narzędzia \texttt{curl}, biblioteki \texttt{requests} w Pythonie i funkcji \texttt{fetch} w JavaScript.
    \item Podstawowe narzędzia DevTools: śledzenie żądań sieciowych i analiza nagłówków.
    \item Co to jest JSON, dlaczego jest popularny i jakie są jego ograniczenia.
  \end{itemize}
\end{frame}

% --- Przykład struktury kolejnych sekcji ---
\section{HTTP — podstawy}

\begin{frame}{Powtórka}
  \begin{itemize}
    \item Serwer to komputer lub program, który udostępnia zasoby lub usługi, a klient wysyła zadania do serwera. Serwerów może być wiele, współpracujących ze sobą.
    \item Frontend to część aplikacji działająca po stronie użytkownika, na przykład przeglądarka renderująca stronę, a backend to serwer odpowiedzialny za przetwarzanie zadań i wysyłanie odpowiedzi.
    \item Współczesne systemy są wielowarstwowe – backend jednej usługi może być frontendem dla innej usługi (mikroserwisy).
    \item Przykład: przeglądarka łączy się z serwerem, który udostępnia pliki HTML, JS, CSS i obrazy; podobnie jądro systemu operacyjnego współpracuje ze sterownikami, które komunikują się z kontrolerami dysków.
  \end{itemize}
\end{frame}

\begin{frame}[fragile]{Jądro — Sterownik — Kontroler dysku}
  \vfill
  \begin{center}
    \begin{minipage}{0.4\textwidth}
      {\scriptsize
      \begin{lstlisting}[style=dark, basicstyle=\ttfamily\scriptsize\color{fgcode}, numbers=none, xleftmargin=1em, xrightmargin=1em]
+---------------------+
|   Jądro (Kernel)    |
|     [Backend]       |
+----------+----------+
           |
           | komunikacja
           |
+----------v---------------+
|    Sterownik (Driver)    |
| [Frontend dla kontrolera |
|   oraz backend dla k.'a] |
+----------+---------------+
           |
           | komunikacja
           |
+----------v----------+
| Kontroler dysku     |
|  (Disk Controller)  |
|     [Hardware]      |
+---------------------+
      \end{lstlisting}
      }
    \end{minipage}
  \end{center}
  \vfill
\end{frame}


\begin{frame}{Model Klient Serwer}
  \vfill
  {\tiny
  \begin{center}
    \begin{tabular}{p{3cm} p{4.5cm} p{6cm}}
      \textbf{Kategoria} & \textbf{Serwer} & \textbf{Typowy klient / protokół} \\[0.5ex]
      \hline
      \\
      \textbf{Pliki i udziały} &
        \textbf{SMB/CIFS} (Samba, Windows File Server) &
        Eksplorator Windows, \texttt{smbclient}, macOS Finder \\[0.5ex]
       &
        \textbf{FTP/SFTP} &
        FileZilla, WinSCP, \texttt{ftp} / \texttt{sftp} w terminalu \\[1ex]
      \textbf{Bazy danych} &
        \textbf{PostgreSQL} &
        psql, pgAdmin, aplikacje korzystające z drivera PG \\[0.5ex]
       &
        \textbf{MongoDB} &
        mongo shell, aplikacje z driverem Mongo \\[1ex]
      \textbf{Web / API} &
        \textbf{HTTP serwer} (Apache, Nginx, Node.js) &
        Przeglądarka, \texttt{curl}, aplikacje mobilne/SPA \\[1ex]
      \textbf{Poczta elektroniczna} &
        \textbf{SMTP} (wysyłka) &
        Mail Transfer Agent (Postfix) → serwery dalej / klient wysyłający \\[0.5ex]
       &
        \textbf{IMAP / POP3} (odbiór) &
        Thunderbird, Outlook, aplikacje mobilne \\[1ex]
      \textbf{Nazwa hostów} &
        \textbf{DNS} &
        System resolver, \texttt{dig}, przeglądarka \\[1ex]
      \textbf{Czas} &
        \textbf{NTP} &
        Usługa czasu w systemie operacyjnym \\[1ex]
      \textbf{Zdalny dostęp} &
        \textbf{SSH} &
        \texttt{ssh} w terminalu, PuTTY, scp/rsync \\[0.5ex]
       &
        \textbf{RDP} &
        Microsoft Remote Desktop, rdesktop \\[1ex]
      \textbf{Kolejki wiadomości / IoT} &
        \textbf{RabbitMQ (AMQP)} &
        Producent/Konsument AMQP, micro-services \\[0.5ex]
       &
        \textbf{MQTT broker} (Mosquitto, HiveMQ) &
        Czujniki IoT, Home-Assistant \\[1ex]
      \textbf{Cache / pamięć klucz-wartość} &
        \textbf{Redis} &
        Aplikacje web, CLI \texttt{redis-cli} \\[1ex]
      \textbf{Druk} &
        \textbf{CUPS Print Server} &
        Dialog drukowania w systemie, IPP \\[1ex]
      \textbf{Strumieniowanie multimediów} &
        \textbf{Icecast / RTSP} &
        VLC, odtwarzacze sieciowe \\[1ex]
      \textbf{Gry online} &
        \textbf{Minecraft Server, CS:GO Dedicated Server} &
        Klient gry \\
    \end{tabular}
  \end{center}

  W każdym z przypadków jedna strona (\textbf{serwer}) \textbf{nasłuchuje} na określonym porcie, czeka na żądania i zarządza współdzielonym zasobem (pliki, dane, czas, druk, audio itp.). Druga strona (\textbf{klient}) \textbf{nawiązuje połączenie}, wysyła zapytania lub komendy i odbiera odpowiedzi. Układ jest więc asymetryczny – serwer utrzymuje usługę, klient z niej korzysta.
  }
  \vfill
\end{frame}

\begin{frame}{Jak działa klasyczny serwer?}
\small
  Serwer odczytuje z dysku plik HTML (oraz powiązane CSS/JS) i przekazuje go \emph{bez zmian} do przeglądarki.

  \vspace{1ex}
  \textbf{Dynamiczna strona (szablony)}
  \begin{itemize}

    \item Szablon HTML zawiera pola do uzupełnienia.
    \item Serwer (lub interpreter / framework) wypełnia je danymi (użytkownik, stan aplikacji, baza).
    \item Powstaje \textbf{nowo wygenerowany} dokument, który trafia do klienta.
  \end{itemize}

  \vspace{1ex}
  \textbf{Techniczny szczegół – odpowiedź HTTP}
  \begin{itemize}
    \item Serwer \textbf{nie wysyła „pliku” jako pliku}, lecz ramkę \textbf{Odpowiedź HTTP}:
      \begin{itemize}\small
        \item nagłówki (status, typ MIME, długość, kompresja)
        \item ciało z treścią żądanego zasobu
      \end{itemize}
    \item Przy statycznej stronie treść = zawartość pliku; przy kompresji (gzip/br) treść jest skompresowana, ale efekt wizualny w przeglądarce pozostaje ten sam.
  \end{itemize}

  \vspace{1ex}
  \textbf{Co decyduje o rodzaju odpowiedzi?}
  \begin{itemize}
    \item \textbf{Nginx w trybie „static”} → serwuje pliki.
    \item \textbf{Apache + PHP-FPM} → uruchamia kod PHP, wynik trafia do klienta.
    \item \textbf{Serwer Pythona (Django/Flask/FastAPI)} → wykonuje kod Pythona i zwraca rezultat (HTML, JSON, inne).
  \end{itemize}
\end{frame}


\begin{frame}{HTTP}
  \begin{itemize}
    \item Protokół warstwy aplikacji odpowiedzialny za komunikację \textbf{klient $\leftrightarrow$ serwer}
    \item Definiuje \textbf{formatowanie i transmisję} wiadomości (żądanie / odpowiedź)
    \item \textbf{Powszechnie stosowany} standard Web (HTTP/1.1, HTTP/2, HTTP/3)
    \item \textbf{Nazwa domeny $\neq$ serwer} – musi zostać przetłumaczona przez \textbf{DNS} na adres IP serwera
  \end{itemize}
  
\end{frame}


\begin{frame}{Metody HTTP}
Żeby zarządzać zasobami stosuje się metody:
  \vspace{1ex}

  \begin{itemize}
    \item \textbf{POST} służy do tworzenia nowych zasobów (Create)
    \item \textbf{GET} służy do pobierania zasobów (Retrieve)
    \item \textbf{PUT} służy do aktualizowania istniejących zasobów (Update)
    \item \textbf{DELETE} służy do usuwania zasobów (Delete)
    \item Oprócz CRUD istnieją też inne metody, np.\ \textbf{HEAD}, \textbf{OPTIONS} czy \textbf{PATCH}.
  \end{itemize}
 
\end{frame}

\begin{frame}{Kody Statusu}

  \begin{itemize}
    \item Teoria mówi, że kody 1xx–5xx jasno informują o wyniku żądania.
    \item W praktyce wiele API zwraca 200 nawet przy błędzie, bo nie obsługują wyjątków w endpointach.
    \item Często domyślny kod 500 pochodzi z frameworka, chociaż lepiej byłoby zwrócić np.\ 404 lub inny odpowiedni kod.
    \item Niektórzy używają POST zamiast GET czy DELETE, bo chcą przesłać więcej danych lub uniknąć ograniczeń GET.
  \end{itemize}
\end{frame}


\begin{frame}[fragile]{Żądanie (Request)}
  \begin{itemize}
    \item Klient wysyła żądanie przez sieć do serwera.
    \item URL wskazuje, jaki zasób chcemy pobrać lub zmodyfikować (może zawierać parametry zapytania).
    \item Metoda HTTP (GET, POST, PUT, DELETE) określa, co ma się stać z zasobem.
  \end{itemize}

  \vspace{1ex}
  \textbf{Podstawowe elementy żądania:}
  \begin{center}
    \begin{minipage}{0.6\textwidth}
      \begin{lstlisting}[style=dark, basicstyle=\ttfamily\scriptsize\color{fgcode}, numbers=none, xleftmargin=1em, xrightmargin=1em]
HTTP REQUEST  (klient -> serwer)
+----------------------------------------+
| Request-Line                           |
|   GET /index.html HTTP/1.1             |
+----------------------------------------+
| Nagłówki (Headers)                     |
|   Host: example.com                    |
|   User-Agent: Mozilla/5.0 ...          |
|   Accept: text/html                    |
|   ...                                  |
+----------------------------------------+
| Pusta linia (CRLF)                     |
+----------------------------------------+
| Body (opcjonalny)                      |
|   <- dla POST/PUT: JSON, formularz,    |
|      plik ->                           |
+----------------------------------------+
      \end{lstlisting}
    \end{minipage}
  \end{center}
 
\end{frame}

\begin{frame}[fragile]{Nagłówki i ciało żądania}
  \begin{itemize}
    \item metoda żądania (np.\ \texttt{POST}), adres URL (np.\ \texttt{/api/users}), wersja HTTP (np.\ \texttt{HTTP/1.1})
    \item \texttt{Host}, czyli docelowy serwer (np.\ \texttt{example.com})
    \item \texttt{Content-Type}, określający format ciała (np.\ \texttt{application/json})
    \item \texttt{Content-Length}, czyli rozmiar ciała żądania
  \end{itemize}

  \vspace{1ex}
  \textbf{W ciele żądania przesyłamy dane (najczęściej w JSON):}
  \begin{center}
    \begin{minipage}{0.6\textwidth}
      \begin{lstlisting}[style=dark, basicstyle=\ttfamily\scriptsize\color{fgcode}, numbers=none, xleftmargin=1em, xrightmargin=1em]
{
    "username": "jan_kowalski",
    "email": "jan_kowalski@example.com",
    "password": "haslo123"
}
      \end{lstlisting}
    \end{minipage}
  \end{center}
\end{frame}

\begin{frame}[fragile]{Odpowiedź (Response)}
  \begin{itemize}
    \item Serwer wysyła odpowiedź do klienta, używając protokołu HTTP.
    \item Status-Line informuje o wersji HTTP, kodzie statusu i krótkim opisie (np.\ \texttt{HTTP/1.1 200 OK}).
    \item Nagłówki (Headers) zawierają meta-informacje o odpowiedzi, takie jak \texttt{Content-Type}, \texttt{Content-Length}, \texttt{Cache-Control} itp.
    \item Po pustej linii (CRLF) pojawia się ciało odpowiedzi (Body), które może być statycznym plikiem (HTML, obraz, CSS) lub wygenerowaną dynamicznie treścią.
  \end{itemize}

  \vspace{1ex}
  \textbf{Przykład struktury odpowiedzi HTTP:}
  \begin{center}
    \begin{minipage}{0.6\textwidth}
      \begin{lstlisting}[style=dark, basicstyle=\ttfamily\scriptsize\color{fgcode}, numbers=none, xleftmargin=1em, xrightmargin=1em]
HTTP RESPONSE  (serwer -> klient)
+------------------------------------------+
| Status-Line                              |
|   HTTP/1.1 200 OK                        |
+------------------------------------------+
| Nagłówki (Headers)                       |
|   Content-Type: text/html; charset=utf-8 |
|   Content-Length: 347                    |
|   Cache-Control: max-age=3600            |
|   ...                                    |
+------------------------------------------+
| Pusta linia (CRLF)                       |
+------------------------------------------+
| Body                                     |
|   <html> ...                             |
+------------------------------------------+
      \end{lstlisting}
    \end{minipage}
  \end{center}

\end{frame}

\begin{frame}[fragile]{Jak wysłać żądania?}
  Możesz użyć narzędzia \texttt{curl} w terminalu, które ma wiele opcji konfiguracyjnych. Przykład prostego żądania \texttt{POST} z JSON-em:
    \vspace{1ex}

  \begin{lstlisting}[style=dark, language=bash, basicstyle=\ttfamily\scriptsize\color{fgcode}, numbers=none, xleftmargin=1em, xrightmargin=1em]
curl -X POST https://twoj-serwer.pl/api/sciezka \
  -H "Content-Type: application/json" \
  -d '{"klucz1":"wartość1","klucz2":123,"klucz3":true}'
  \end{lstlisting}
    \vspace{1ex}

  DevTools w przeglądarkach (Chrome, Firefox) to zintegrowane narzędzia dla deweloperów, które ułatwiają:
  \begin{itemize}
    \item analizę struktury i stylów strony,
    \item podgląd i debugowanie kodu JavaScript,
    \item monitorowanie sieci (\texttt{Network}) — widać wszystkie wysyłane żądania i odpowiedzi,
    \item zarządzanie pamięcią podręczną, profilowanie wydajności i inne.
  \end{itemize}
\end{frame}


\begin{frame}[fragile]{fetch (JavaScript)}
W JavaScript możesz użyć funkcji `fetch`:

  \begin{lstlisting}[style=dark, language=JavaScript, basicstyle=\ttfamily\scriptsize\color{fgcode}, numbers=none, xleftmargin=1em, xrightmargin=1em]
fetch('https://twoj-serwer.pl/api/sciezka', {
  method: 'POST',
  headers: {
    'Content-Type': 'application/json'
  },
  body: JSON.stringify({
    klucz1: 'wartość1',
    klucz2: 123,
    klucz3: true
  })
})
.then(response => {
  if (!response.ok) {
    throw new Error(`Błąd sieci: ${response.status}`);
  }
  return response.json();
})
.then(data => {
  console.log('Odpowiedź z serwera:', data);
})
.catch(error => {
  console.error('Coś poszło nie tak:', error);
});
  \end{lstlisting}
  
\end{frame}

\begin{frame}[fragile]{requests (Python)}
W Pythonie, za pomocą biblioteki `requests`, wygląda to tak:

  \begin{lstlisting}[style=dark, language=Python, basicstyle=\ttfamily\scriptsize\color{fgcode}, numbers=none, xleftmargin=1em, xrightmargin=1em]
import requests

url = 'https://twoj-serwer.pl/api/sciezka'
payload = {
  "klucz1": "wartość1",
  "klucz2": 123,
  "klucz3": True
}
response = requests.post(url, json=payload)
if response.status_code == 200:
  data = response.json()
  print('Odpowiedź z serwera:', data)
else:
  print('Błąd:', response.status_code)
  \end{lstlisting}
  
\end{frame}

\begin{frame}[fragile]{Przykład żądania GET}
  \begin{itemize}
    \item \texttt{http://wttr.in/NAZWA\_MIASTA}
    \item Chcemy pobrać dane pogodowe dla konkretnego miasta, np.\ Warszawy.
    \item Nazwa hosta: \texttt{wttr.in}
    \item Ścieżka: \texttt{/Warsaw} (gdzie \texttt{Warsaw} to \texttt{/NAZWA\_MIASTA})
  \end{itemize}

  \vspace{1ex}
  \textbf{Połączenie TCP/DNS}

  Przeglądarka/klient tłumaczy \texttt{wttr.in} na adres IP i otwiera połączenie na porcie 80.

  \vspace{1ex}
  \textbf{Żądanie HTTP:}
  \begin{center}
    \begin{minipage}{0.5\textwidth}
      \begin{lstlisting}[style=dark, basicstyle=\ttfamily\scriptsize\color{fgcode}, numbers=none, xleftmargin=1em, xrightmargin=1em]
GET /Warsaw HTTP/1.1
Host: wttr.in
Accept: application/json
Connection: close
      \end{lstlisting}
    \end{minipage}
  \end{center}

\end{frame}

\begin{frame}[fragile]{Odpowiedź HTTP}
  \begin{lstlisting}[style=dark, basicstyle=\ttfamily\scriptsize\color{fgcode}, numbers=none, xleftmargin=1em, xrightmargin=1em]
HTTP/1.1 200 OK
Content-Type: application/json; charset=utf-8
Content-Length: 1024
Cache-Control: max-age=600
Connection: close

{
 "current_condition":[
   {
     "temp_C":"16",
     "weatherDesc":[{"value":"Sunny"}],
     ...
   }
 ],
 "weather":[
   {
     "date":"2025-06-03",
     ...
   },
   ...
 ]
}
  \end{lstlisting}

  \vspace{1ex}
  \begin{itemize}
    \item Ścieżka \texttt{/Berlin} to \texttt{/NAZWA\_MIASTA}.
    \item Nagłówek \texttt{Accept: application/json} prosi o JSON.
    \item Serwer zwraca kod 200, nagłówki z typem treści \texttt{application/json} i ciało z danymi pogodowymi.
  \end{itemize}
\end{frame}

\begin{frame}[fragile]{Przykład żądania POST}
Wyślijmy żądanie POST na adres  \texttt{jsonplaceholder.typicode.com/posts}.
  \begin{center}
    \begin{minipage}{0.5\textwidth}
      \begin{lstlisting}[style=dark, basicstyle=\ttfamily\scriptsize\color{fgcode}, numbers=none, xleftmargin=1em, xrightmargin=1em]
POST /posts HTTP/1.1
Host: jsonplaceholder.typicode.com
Content-Type: application/json; charset=UTF-8
Content-Length:  fifty

{"title":"abc","body":"xyz","userId":5}
      \end{lstlisting}
    \end{minipage}
  \end{center}

  \vspace{1ex}
  \begin{itemize}
    \item Metoda: \textbf{POST}
    \item Ścieżka: \texttt{/posts}
    \item Ciało żądania (JSON): \texttt{\{"title":"abc","body":"xyz","userId":5\}}
  \end{itemize}
\end{frame}

\begin{frame}[fragile]{Przykład w \texttt{curl}}
  \begin{lstlisting}[style=dark, language=bash, basicstyle=\ttfamily\scriptsize\color{fgcode}, numbers=none, xleftmargin=1em, xrightmargin=1em]
curl -X POST https://jsonplaceholder.typicode.com/posts \
     -H "Content-Type: application/json; charset=UTF-8" \
     -d '{"title":"abc","body":"xyz","userId":5}'
  \end{lstlisting}


  \vspace{1ex}
  \textbf{Odpowiedź (JSON):}
  \begin{verbatim}
{
  "id": 101,
  "title": "abc",
  "body": "xyz",
  "userId": 5
}
  \end{verbatim}
\end{frame}


\begin{frame}{API}
  \begin{itemize}
    \item API (Application Programming Interface) to zestaw reguł i protokołów umożliwiający komunikację między aplikacjami, niekoniecznie tylko przez internet.
    \item Przykładem może być API Windows, które pozwala na dostęp do funkcji systemowych z poziomu innej aplikacji.
    \item API dostarcza interfejs programistyczny, dzięki któremu programiści mogą korzystać z funkcjonalności innych programów lub usług.
    \item W ten sposób można tworzyć nowe aplikacje, wykorzystując już istniejące zasoby — na przykład pobierać dane lub wywoływać usługi zdalne.
    \item API pozwala na współdzielenie danych i funkcji między różnymi systemami, co ułatwia integrację i budowanie bardziej zaawansowanych rozwiązań.
  \end{itemize}

  \vspace{1ex}
  \textbf{Przykłady API działających przez HTTP:}
  \begin{itemize}
    \item \url{https://jsonplaceholder.typicode.com/}
    \item \url{https://api.github.com}
  \end{itemize}
\end{frame}

\begin{frame}[fragile]{Budujemy własne API (Flask)}
  \begin{itemize}
    \item Jeden adres zasobu – różne metody (GET, POST, PUT, DELETE…)
    \item Kod stanu HTTP opisuje wynik (200, 201, 404…)
    \item JSON to standardowy format danych (czytelny dla ludzi i maszyn)
  \end{itemize}

  \vspace{1ex}
  \begin{lstlisting}[style=dark, language=python, basicstyle=\ttfamily\scriptsize\color{fgcode}, numbers=none, xleftmargin=1em, xrightmargin=1em]
ksiazki = [
    {"id": 1, "tytul": "Przygody Tomka Sawyera", "autor": "Mark Twain"},
    {"id": 2, "tytul": "Władca Pierścieni", "autor": "J.R.R. Tolkien"},
    {"id": 3, "tytul": "Ojciec Chrzestny", "autor": "Mario Puzo"}
]
  \end{lstlisting}
\end{frame}

\begin{frame}[fragile]{Endpoint GET – lista książek}
  \begin{lstlisting}[style=dark, language=python, basicstyle=\ttfamily\scriptsize\color{fgcode}, numbers=none, xleftmargin=1em, xrightmargin=1em]
@app.route("/ksiazki")
def lista_ksiazek():
    return jsonify(ksiazki)          # 200 OK
  \end{lstlisting}

  \vspace{1ex}
  \begin{itemize}
    \item Żądanie: \texttt{GET /ksiazki}
    \item Odpowiedź: \texttt{200 OK} + JSON z tablicą książek
  \end{itemize}
\end{frame}


\begin{frame}[fragile]{Endpoint POST – dodawanie książki}
  \begin{lstlisting}[style=dark, language=python, basicstyle=\ttfamily\scriptsize\color{fgcode}, numbers=none, xleftmargin=1em, xrightmargin=1em]
@app.route("/ksiazki", methods=["POST"])
def dodaj_ksiazke():
    nowa = request.json              # oczekuje {"id": 4, "tytul": "...", "autor": "..."}
    ksiazki.append(nowa)
    return jsonify(nowa), 201        # 201 Created
  \end{lstlisting}

  \vspace{1ex}
  \begin{itemize}
    \item Żądanie: \texttt{POST /ksiazki} + nagłówek \texttt{Content-Type: application/json}
    \item Odpowiedź: \texttt{201 Created} + JSON nowo dodanego rekordu
  \end{itemize}
\end{frame}

\begin{frame}{Protokoły aplikacyjne / Style interfejsów API}
  \begin{itemize}
    \item HTTP to protokół transportowy używany do przesyłania żądań i odpowiedzi; REST to styl architektury, który korzysta z HTTP do operacji na zasobach
    \item REST API najczęściej zwraca JSON (dawniej bywał XML), wykorzystuje standardowe metody HTTP (GET, POST, PUT, DELETE) i kody statusu
    \item GraphQL to język zapytań zbudowany nad HTTP (można go kapsułkować w POST), pozwala klientowi precyzyjnie określić, jakich danych potrzebuje
    \item MQTT to lekki protokół publikuj–subskrybuj, używany głównie w IoT i przesyłaniu strumieniowym danych, nie jest typowym API-em REST-owym
    \item WebSockets oferują dwukierunkową, utrzymywaną na stałe sesję TCP, często stosowane do czatu, powiadomień lub strumieniowania (np. wraz z MQTT)
    \item SOAP to starszy protokół XML-owy oparty na wiadomościach SOAP, dziś rzadko spotykany (pozostał głównie w starszych systemach korporacyjnych)
  \end{itemize}
\end{frame}


\end{document}
